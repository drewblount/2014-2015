\chapter*{Abstract}

This thesis discusses a general method for the global optimization of expensive blackbox functions: sequential model-based optimization. `SMBO' works by iteratively fitting a prediction model to available data, and analyzing that model to generate another model with greater predictive power. By recursively iterating this process, using successive prediction models to bootstrap ever-better models, very good solutions are found to NP-hard optimization problems.

This thesis starts in mathematical statistics, and ends in computer science and software design. I will first describe the SMBO process in general: its history, applications, and relevance to optimization today. I will then rigorously present a paradigmatic example of SMBO, the `EGO Algorithm' of Donald Jones et al. A major goal of this thesis is to build a robust implementation of sequential model-based optimization capable of optimizing real-world processes. Thus I also include a detailed methods section presenting my Python package, $smbo$, a modular SMBO framework. With this program, different SMBO processes are implemented easily by providing functional components to a general optimizer object. The EGO Algorithm is then applied to a graphical and emergent system in Craig Reynold's Boids, where I use it to `hyper-evolve' the behavior of a simulated flock of birds.