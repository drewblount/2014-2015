% Generated by Sphinx.
\def\sphinxdocclass{report}
\documentclass[letterpaper,10pt,english]{sphinxmanual}
\usepackage[utf8]{inputenc}
\DeclareUnicodeCharacter{00A0}{\nobreakspace}
\usepackage{cmap}
\usepackage[T1]{fontenc}
\usepackage{babel}
\usepackage{times}
\usepackage[Bjarne]{fncychap}
\usepackage{longtable}
\usepackage{sphinx}
\usepackage{multirow}

\addto\captionsenglish{\renewcommand{\figurename}{Fig. }}
\addto\captionsenglish{\renewcommand{\tablename}{Table }}
\floatname{literal-block}{Listing }



\title{smbo Documentation}
\date{April 21, 2015}
\release{0.1}
\author{Drew Blount}
\newcommand{\sphinxlogo}{}
\renewcommand{\releasename}{Release}
\makeindex

\makeatletter
\def\PYG@reset{\let\PYG@it=\relax \let\PYG@bf=\relax%
    \let\PYG@ul=\relax \let\PYG@tc=\relax%
    \let\PYG@bc=\relax \let\PYG@ff=\relax}
\def\PYG@tok#1{\csname PYG@tok@#1\endcsname}
\def\PYG@toks#1+{\ifx\relax#1\empty\else%
    \PYG@tok{#1}\expandafter\PYG@toks\fi}
\def\PYG@do#1{\PYG@bc{\PYG@tc{\PYG@ul{%
    \PYG@it{\PYG@bf{\PYG@ff{#1}}}}}}}
\def\PYG#1#2{\PYG@reset\PYG@toks#1+\relax+\PYG@do{#2}}

\expandafter\def\csname PYG@tok@gd\endcsname{\def\PYG@tc##1{\textcolor[rgb]{0.63,0.00,0.00}{##1}}}
\expandafter\def\csname PYG@tok@gu\endcsname{\let\PYG@bf=\textbf\def\PYG@tc##1{\textcolor[rgb]{0.50,0.00,0.50}{##1}}}
\expandafter\def\csname PYG@tok@gt\endcsname{\def\PYG@tc##1{\textcolor[rgb]{0.00,0.27,0.87}{##1}}}
\expandafter\def\csname PYG@tok@gs\endcsname{\let\PYG@bf=\textbf}
\expandafter\def\csname PYG@tok@gr\endcsname{\def\PYG@tc##1{\textcolor[rgb]{1.00,0.00,0.00}{##1}}}
\expandafter\def\csname PYG@tok@cm\endcsname{\let\PYG@it=\textit\def\PYG@tc##1{\textcolor[rgb]{0.25,0.50,0.56}{##1}}}
\expandafter\def\csname PYG@tok@vg\endcsname{\def\PYG@tc##1{\textcolor[rgb]{0.73,0.38,0.84}{##1}}}
\expandafter\def\csname PYG@tok@m\endcsname{\def\PYG@tc##1{\textcolor[rgb]{0.13,0.50,0.31}{##1}}}
\expandafter\def\csname PYG@tok@mh\endcsname{\def\PYG@tc##1{\textcolor[rgb]{0.13,0.50,0.31}{##1}}}
\expandafter\def\csname PYG@tok@cs\endcsname{\def\PYG@tc##1{\textcolor[rgb]{0.25,0.50,0.56}{##1}}\def\PYG@bc##1{\setlength{\fboxsep}{0pt}\colorbox[rgb]{1.00,0.94,0.94}{\strut ##1}}}
\expandafter\def\csname PYG@tok@ge\endcsname{\let\PYG@it=\textit}
\expandafter\def\csname PYG@tok@vc\endcsname{\def\PYG@tc##1{\textcolor[rgb]{0.73,0.38,0.84}{##1}}}
\expandafter\def\csname PYG@tok@il\endcsname{\def\PYG@tc##1{\textcolor[rgb]{0.13,0.50,0.31}{##1}}}
\expandafter\def\csname PYG@tok@go\endcsname{\def\PYG@tc##1{\textcolor[rgb]{0.20,0.20,0.20}{##1}}}
\expandafter\def\csname PYG@tok@cp\endcsname{\def\PYG@tc##1{\textcolor[rgb]{0.00,0.44,0.13}{##1}}}
\expandafter\def\csname PYG@tok@gi\endcsname{\def\PYG@tc##1{\textcolor[rgb]{0.00,0.63,0.00}{##1}}}
\expandafter\def\csname PYG@tok@gh\endcsname{\let\PYG@bf=\textbf\def\PYG@tc##1{\textcolor[rgb]{0.00,0.00,0.50}{##1}}}
\expandafter\def\csname PYG@tok@ni\endcsname{\let\PYG@bf=\textbf\def\PYG@tc##1{\textcolor[rgb]{0.84,0.33,0.22}{##1}}}
\expandafter\def\csname PYG@tok@nl\endcsname{\let\PYG@bf=\textbf\def\PYG@tc##1{\textcolor[rgb]{0.00,0.13,0.44}{##1}}}
\expandafter\def\csname PYG@tok@nn\endcsname{\let\PYG@bf=\textbf\def\PYG@tc##1{\textcolor[rgb]{0.05,0.52,0.71}{##1}}}
\expandafter\def\csname PYG@tok@no\endcsname{\def\PYG@tc##1{\textcolor[rgb]{0.38,0.68,0.84}{##1}}}
\expandafter\def\csname PYG@tok@na\endcsname{\def\PYG@tc##1{\textcolor[rgb]{0.25,0.44,0.63}{##1}}}
\expandafter\def\csname PYG@tok@nb\endcsname{\def\PYG@tc##1{\textcolor[rgb]{0.00,0.44,0.13}{##1}}}
\expandafter\def\csname PYG@tok@nc\endcsname{\let\PYG@bf=\textbf\def\PYG@tc##1{\textcolor[rgb]{0.05,0.52,0.71}{##1}}}
\expandafter\def\csname PYG@tok@nd\endcsname{\let\PYG@bf=\textbf\def\PYG@tc##1{\textcolor[rgb]{0.33,0.33,0.33}{##1}}}
\expandafter\def\csname PYG@tok@ne\endcsname{\def\PYG@tc##1{\textcolor[rgb]{0.00,0.44,0.13}{##1}}}
\expandafter\def\csname PYG@tok@nf\endcsname{\def\PYG@tc##1{\textcolor[rgb]{0.02,0.16,0.49}{##1}}}
\expandafter\def\csname PYG@tok@si\endcsname{\let\PYG@it=\textit\def\PYG@tc##1{\textcolor[rgb]{0.44,0.63,0.82}{##1}}}
\expandafter\def\csname PYG@tok@s2\endcsname{\def\PYG@tc##1{\textcolor[rgb]{0.25,0.44,0.63}{##1}}}
\expandafter\def\csname PYG@tok@vi\endcsname{\def\PYG@tc##1{\textcolor[rgb]{0.73,0.38,0.84}{##1}}}
\expandafter\def\csname PYG@tok@nt\endcsname{\let\PYG@bf=\textbf\def\PYG@tc##1{\textcolor[rgb]{0.02,0.16,0.45}{##1}}}
\expandafter\def\csname PYG@tok@nv\endcsname{\def\PYG@tc##1{\textcolor[rgb]{0.73,0.38,0.84}{##1}}}
\expandafter\def\csname PYG@tok@s1\endcsname{\def\PYG@tc##1{\textcolor[rgb]{0.25,0.44,0.63}{##1}}}
\expandafter\def\csname PYG@tok@gp\endcsname{\let\PYG@bf=\textbf\def\PYG@tc##1{\textcolor[rgb]{0.78,0.36,0.04}{##1}}}
\expandafter\def\csname PYG@tok@sh\endcsname{\def\PYG@tc##1{\textcolor[rgb]{0.25,0.44,0.63}{##1}}}
\expandafter\def\csname PYG@tok@ow\endcsname{\let\PYG@bf=\textbf\def\PYG@tc##1{\textcolor[rgb]{0.00,0.44,0.13}{##1}}}
\expandafter\def\csname PYG@tok@sx\endcsname{\def\PYG@tc##1{\textcolor[rgb]{0.78,0.36,0.04}{##1}}}
\expandafter\def\csname PYG@tok@bp\endcsname{\def\PYG@tc##1{\textcolor[rgb]{0.00,0.44,0.13}{##1}}}
\expandafter\def\csname PYG@tok@c1\endcsname{\let\PYG@it=\textit\def\PYG@tc##1{\textcolor[rgb]{0.25,0.50,0.56}{##1}}}
\expandafter\def\csname PYG@tok@kc\endcsname{\let\PYG@bf=\textbf\def\PYG@tc##1{\textcolor[rgb]{0.00,0.44,0.13}{##1}}}
\expandafter\def\csname PYG@tok@c\endcsname{\let\PYG@it=\textit\def\PYG@tc##1{\textcolor[rgb]{0.25,0.50,0.56}{##1}}}
\expandafter\def\csname PYG@tok@mf\endcsname{\def\PYG@tc##1{\textcolor[rgb]{0.13,0.50,0.31}{##1}}}
\expandafter\def\csname PYG@tok@err\endcsname{\def\PYG@bc##1{\setlength{\fboxsep}{0pt}\fcolorbox[rgb]{1.00,0.00,0.00}{1,1,1}{\strut ##1}}}
\expandafter\def\csname PYG@tok@mb\endcsname{\def\PYG@tc##1{\textcolor[rgb]{0.13,0.50,0.31}{##1}}}
\expandafter\def\csname PYG@tok@ss\endcsname{\def\PYG@tc##1{\textcolor[rgb]{0.32,0.47,0.09}{##1}}}
\expandafter\def\csname PYG@tok@sr\endcsname{\def\PYG@tc##1{\textcolor[rgb]{0.14,0.33,0.53}{##1}}}
\expandafter\def\csname PYG@tok@mo\endcsname{\def\PYG@tc##1{\textcolor[rgb]{0.13,0.50,0.31}{##1}}}
\expandafter\def\csname PYG@tok@kd\endcsname{\let\PYG@bf=\textbf\def\PYG@tc##1{\textcolor[rgb]{0.00,0.44,0.13}{##1}}}
\expandafter\def\csname PYG@tok@mi\endcsname{\def\PYG@tc##1{\textcolor[rgb]{0.13,0.50,0.31}{##1}}}
\expandafter\def\csname PYG@tok@kn\endcsname{\let\PYG@bf=\textbf\def\PYG@tc##1{\textcolor[rgb]{0.00,0.44,0.13}{##1}}}
\expandafter\def\csname PYG@tok@o\endcsname{\def\PYG@tc##1{\textcolor[rgb]{0.40,0.40,0.40}{##1}}}
\expandafter\def\csname PYG@tok@kr\endcsname{\let\PYG@bf=\textbf\def\PYG@tc##1{\textcolor[rgb]{0.00,0.44,0.13}{##1}}}
\expandafter\def\csname PYG@tok@s\endcsname{\def\PYG@tc##1{\textcolor[rgb]{0.25,0.44,0.63}{##1}}}
\expandafter\def\csname PYG@tok@kp\endcsname{\def\PYG@tc##1{\textcolor[rgb]{0.00,0.44,0.13}{##1}}}
\expandafter\def\csname PYG@tok@w\endcsname{\def\PYG@tc##1{\textcolor[rgb]{0.73,0.73,0.73}{##1}}}
\expandafter\def\csname PYG@tok@kt\endcsname{\def\PYG@tc##1{\textcolor[rgb]{0.56,0.13,0.00}{##1}}}
\expandafter\def\csname PYG@tok@sc\endcsname{\def\PYG@tc##1{\textcolor[rgb]{0.25,0.44,0.63}{##1}}}
\expandafter\def\csname PYG@tok@sb\endcsname{\def\PYG@tc##1{\textcolor[rgb]{0.25,0.44,0.63}{##1}}}
\expandafter\def\csname PYG@tok@k\endcsname{\let\PYG@bf=\textbf\def\PYG@tc##1{\textcolor[rgb]{0.00,0.44,0.13}{##1}}}
\expandafter\def\csname PYG@tok@se\endcsname{\let\PYG@bf=\textbf\def\PYG@tc##1{\textcolor[rgb]{0.25,0.44,0.63}{##1}}}
\expandafter\def\csname PYG@tok@sd\endcsname{\let\PYG@it=\textit\def\PYG@tc##1{\textcolor[rgb]{0.25,0.44,0.63}{##1}}}

\def\PYGZbs{\char`\\}
\def\PYGZus{\char`\_}
\def\PYGZob{\char`\{}
\def\PYGZcb{\char`\}}
\def\PYGZca{\char`\^}
\def\PYGZam{\char`\&}
\def\PYGZlt{\char`\<}
\def\PYGZgt{\char`\>}
\def\PYGZsh{\char`\#}
\def\PYGZpc{\char`\%}
\def\PYGZdl{\char`\$}
\def\PYGZhy{\char`\-}
\def\PYGZsq{\char`\'}
\def\PYGZdq{\char`\"}
\def\PYGZti{\char`\~}
% for compatibility with earlier versions
\def\PYGZat{@}
\def\PYGZlb{[}
\def\PYGZrb{]}
\makeatother

\renewcommand\PYGZsq{\textquotesingle}

\begin{document}

\maketitle
\tableofcontents
\phantomsection\label{index::doc}


built as part of Drew Blount's Senior Mathematics Thesis at Reed College
\phantomsection\label{index:module-smbo}\index{smbo (module)}
..Indices and tables
..==================

..* \DUspan{xref,std,std-ref}{genindex}
..* \DUspan{xref,std,std-ref}{modindex}
..* \DUspan{xref,std,std-ref}{search}

..Installation
..------------

..Install smbo by running:

..Source
..----------

..https://github.com/drewblount/2014-2015/tree/master/thesis/code/smbo


\chapter{Documentation for the Code}
\label{index:module-smbo.smb_optimizer}\label{index:smbo-a-tool-for-sequential-model-based-optimization}\label{index:documentation-for-the-code}\index{smbo.smb\_optimizer (module)}\phantomsection\label{index:module-smb_optimizer}\index{smb\_optimizer (module)}\index{smb\_optimizer (class in smbo.smb\_optimizer)}

\begin{fulllineitems}
\phantomsection\label{index:smbo.smb_optimizer.smb_optimizer}\pysiglinewithargsret{\strong{class }\code{smbo.smb\_optimizer.}\bfcode{smb\_optimizer}}{\emph{domain}, \emph{objective\_func}, \emph{modeller}, \emph{init\_sampler=None}, \emph{res=0.1}, \emph{brute\_optimize\_EI=False}, \emph{logger=None}}{}
An object that, given an input domain, objective function, and modelling strategy, seeks to efficiently find
the global optimum of the objective by the generation of sequential models.
\begin{quote}\begin{description}
\item[{Parameters}] \leavevmode\begin{itemize}
\item {} 
\textbf{\texttt{domain}} (\emph{list}) -- a \(k\)-list of tuples describing the lower and upper bounds of each input dimension.

\item {} 
\textbf{\texttt{objective\_func}} (\emph{function}) -- a function (or any object with a suitable \_\_apply\_\_ method)
that maps \(k\)-lists to numbers. The goal of an smb\_optimizer is to minimize this
function over the domain.

\item {} 
\textbf{\texttt{modeller}} (\emph{function}) -- a function (or any object with a suitable \_\_apply\_\_ method) that maps a
tuple \((X,Y)\), where \(X\) is a list of sample points (each a \(k\)-vector from the domain),
and \(Y\) their evaluated objective values; to a tuple of functions
\((\hat{y},\ \hat{\sigma}^2)\). These output functions map points in the input domain to real numbers.
\(\hat{y}\) represents the model's best estimate of the objective function, and \(\hat{\sigma}^2\)
is the estimated error of the prediction \(\hat{y}\).

\item {} 
\textbf{\texttt{init\_sampler}} (\emph{function}) -- a function which will select initial sample points, informing the zero-generation model.
If left unspecified, is by default set to a \(2k+2\)-sample latin hypercube over the domain,
created with \code{smbo.latin\_hypercube}.

\item {} 
\textbf{\texttt{res}} (\emph{float}) -- the resolution of any created plots

\item {} 
\textbf{\texttt{brute\_optimize\_EI}} (\emph{bool}) -- if true, the expected improvement function is maximized by actually evaluating it over a plot\_res grid and choosing the argmax. If False, it is maximized by a scipy optimizer. This allows for extremely slow, but trustworthy optimization.

\item {} 
\textbf{\texttt{logger}} -- a python logging.logger object. If none, there is no logging

\item {} 
\textbf{\texttt{X}} (\emph{list}) -- The list of points where \code{objective\_func} has been evaluated already

\item {} 
\textbf{\texttt{Y}} (\emph{list}) -- The list of associated objective function values.

\item {} 
\textbf{\texttt{pred\_y}} (\emph{function), pred\_err (function}) -- The predictor and predicted error surfaces; the output of \code{modeller(X,Y)}.

\end{itemize}

\item[{Returns}] \leavevmode
An smb\_optimizer object.

\end{description}\end{quote}

Attributes:
\index{check\_memory() (smbo.smb\_optimizer.smb\_optimizer method)}

\begin{fulllineitems}
\phantomsection\label{index:smbo.smb_optimizer.smb_optimizer.check_memory}\pysiglinewithargsret{\bfcode{check\_memory}}{\emph{eps=1e-06}}{}~\begin{quote}\begin{description}
\item[{Parameters}] \leavevmode
\textbf{\texttt{eps}} (\emph{float}) -- the margin of error

\item[{Returns}] \leavevmode
checks whether the predictor function correctly interpolates, i.e., if
the predicted function value matches the observed function value at all sample points

\item[{Return type}] \leavevmode
bool

\end{description}\end{quote}

\end{fulllineitems}

\index{dim\_plotter() (smbo.smb\_optimizer.smb\_optimizer method)}

\begin{fulllineitems}
\phantomsection\label{index:smbo.smb_optimizer.smb_optimizer.dim_plotter}\pysiglinewithargsret{\bfcode{dim\_plotter}}{\emph{plot\_dims}}{}
helper function for selecting either plot1d or plot2d

\end{fulllineitems}

\index{exp\_improvement() (smbo.smb\_optimizer.smb\_optimizer method)}

\begin{fulllineitems}
\phantomsection\label{index:smbo.smb_optimizer.smb_optimizer.exp_improvement}\pysiglinewithargsret{\bfcode{exp\_improvement}}{\emph{x\_new}}{}~\begin{quote}\begin{description}
\item[{Returns}] \leavevmode
the expected improvement function evaluated at x\_new

\item[{Return type}] \leavevmode
float

\end{description}\end{quote}

\end{fulllineitems}

\index{plot1d() (smbo.smb\_optimizer.smb\_optimizer method)}

\begin{fulllineitems}
\phantomsection\label{index:smbo.smb_optimizer.smb_optimizer.plot1d}\pysiglinewithargsret{\bfcode{plot1d}}{\emph{dim=0}, \emph{plot\_objective=False}, \emph{plot\_improvement=True}, \emph{plot\_next\_sp=True}, \emph{show\_plot=False}, \emph{fname=None}}{}~\begin{quote}\begin{description}
\item[{Parameters}] \leavevmode\begin{itemize}
\item {} 
\textbf{\texttt{plot\_objective}} (\emph{bool}) -- whether the objective function should be plotted. This is only feasible if that function can be called enough times to generate a smooth plot

\item {} 
\textbf{\texttt{plot\_improvement}} (\emph{bool}) -- option to overlay expected emprovement at each x

\item {} 
\textbf{\texttt{plot\_next\_sp}} (\emph{bool}) -- whether the selection of the next sample point is plotted

\item {} 
\textbf{\texttt{show\_plot}} (\emph{bool}) -- whether pyplot.show is called at the end of the function

\item {} 
\textbf{\texttt{dim}} (\emph{int}) -- the index of the dimension of interest

\item {} 
\textbf{\texttt{x\_delta}} (\emph{float}) -- the resolution of the plot

\item {} 
\textbf{\texttt{fname}} -- if set, the plot is saved to this

\end{itemize}

\end{description}\end{quote}

Uses pyplot to make a nice 1d plot of the predictor function and its error. Optionally, overlay a plot of the

\end{fulllineitems}

\index{sample() (smbo.smb\_optimizer.smb\_optimizer method)}

\begin{fulllineitems}
\phantomsection\label{index:smbo.smb_optimizer.smb_optimizer.sample}\pysiglinewithargsret{\bfcode{sample}}{}{}
Chooses the next sample point by maximizing \code{exp\_improvement}.
Evaluates \code{objective\_func} there, updating \code{X} and \code{Y}. Regenerates predictive models.

\end{fulllineitems}

\index{take\_samples() (smbo.smb\_optimizer.smb\_optimizer method)}

\begin{fulllineitems}
\phantomsection\label{index:smbo.smb_optimizer.smb_optimizer.take_samples}\pysiglinewithargsret{\bfcode{take\_samples}}{\emph{stopping\_improvement=0.01}, \emph{max\_iters=100}, \emph{plot\_dims=None}, \emph{fname='plots/'}, \emph{randomize=False}, \emph{verbose=True}}{}~\begin{quote}\begin{description}
\item[{Parameters}] \leavevmode\begin{itemize}
\item {} 
\textbf{\texttt{stopping\_improvement}} (\emph{float}) -- the iterative process terminates if the maximum expected improvement nowhere is larger than this value

\item {} 
\textbf{\texttt{num\_iters}} (\emph{int}) -- maximum number of successive sample points to be chosen

\item {} 
\textbf{\texttt{plot\_dims}} (\emph{int) or (list}) -- the one, or two dimensions along which plots should be saved.

\item {} 
\textbf{\texttt{fname}} (\emph{string}) -- the prefix of the filename of each file to be saved

\item {} 
\textbf{\texttt{randomize}} (\emph{bool}) -- disabled; being passed along

\end{itemize}

\end{description}\end{quote}

Iteratively chooses a sample point, evaluates the objective function, and refits the model

\end{fulllineitems}


\end{fulllineitems}

\phantomsection\label{index:module-smbo.models}\index{smbo.models (module)}\phantomsection\label{index:module-models}\index{models (module)}\index{dace (class in smbo.models)}

\begin{fulllineitems}
\phantomsection\label{index:smbo.models.dace}\pysiglinewithargsret{\strong{class }\code{smbo.models.}\bfcode{dace}}{\emph{X}, \emph{Y}}{}
A class that implements the DACE model to produce predictor and error surfaces from sample points
\begin{quote}\begin{description}
\item[{Parameters}] \leavevmode\begin{itemize}
\item {} 
\textbf{\texttt{X}} (\emph{list}) -- a list of input vectors

\item {} 
\textbf{\texttt{Y}} (\emph{list}) -- a list of observed objective values

\end{itemize}

\item[{Returns}] \leavevmode
(pred\_y,pred\_err): two functions, each k-to-1, where k is the dimension of the input space, representing the DACE predictor and predicted error at any point in input space.

\item[{Return type}] \leavevmode
tuple

\end{description}\end{quote}
\index{conc\_likelihood() (smbo.models.dace method)}

\begin{fulllineitems}
\phantomsection\label{index:smbo.models.dace.conc_likelihood}\pysiglinewithargsret{\bfcode{conc\_likelihood}}{\emph{new\_P=None}, \emph{new\_Q=None}}{}~\begin{quote}\begin{description}
\item[{Parameters}] \leavevmode\begin{itemize}
\item {} 
\textbf{\texttt{new\_P}} (\emph{list}) -- an \(n\)-vector resetting the \(p\) parameter of the DACE model

\item {} 
\textbf{\texttt{new\_Q}} (\emph{list}) -- an \(n\)-vector resetting the \(q\) or :math:{}`        heta{}` parameter of the DACE model

\end{itemize}

\end{description}\end{quote}

Returns the statistical likelihood of the current DACE parameters \code{P' and :code:{}`Q', given the data :code:{}`X} and \code{Y}.

\end{fulllineitems}

\index{corr() (smbo.models.dace method)}

\begin{fulllineitems}
\phantomsection\label{index:smbo.models.dace.corr}\pysiglinewithargsret{\bfcode{corr}}{\emph{x1}, \emph{x2}}{}~\begin{quote}\begin{description}
\item[{Parameters}] \leavevmode\begin{itemize}
\item {} 
\textbf{\texttt{x1}} (\emph{list}) -- a coordinate in the domain

\item {} 
\textbf{\texttt{x2}} (\emph{list}) -- a coordinate in the domain

\end{itemize}

\item[{Returns}] \leavevmode
the correlation between estimation errors at x1 and x2

\item[{Return type}] \leavevmode
float

\end{description}\end{quote}

\end{fulllineitems}

\index{corr\_vector() (smbo.models.dace method)}

\begin{fulllineitems}
\phantomsection\label{index:smbo.models.dace.corr_vector}\pysiglinewithargsret{\bfcode{corr\_vector}}{\emph{x\_new}}{}~\begin{quote}\begin{description}
\item[{Parameters}] \leavevmode
\textbf{\texttt{x\_new}} -- a \(k\)-vector from the domain

\item[{Returns}] \leavevmode
a vector whose \(i^{th}\) element is the parameterized correlation between x\_new and the :math{}`i\textasciicircum{}\{th\}{}` sample point

\item[{Return type}] \leavevmode
list

\end{description}\end{quote}

\end{fulllineitems}

\index{dist() (smbo.models.dace method)}

\begin{fulllineitems}
\phantomsection\label{index:smbo.models.dace.dist}\pysiglinewithargsret{\bfcode{dist}}{\emph{x1}, \emph{x2}}{}
Returns the parameterized distance between two points in input space

\end{fulllineitems}

\index{exp\_improvement() (smbo.models.dace method)}

\begin{fulllineitems}
\phantomsection\label{index:smbo.models.dace.exp_improvement}\pysiglinewithargsret{\bfcode{exp\_improvement}}{\emph{x\_new}}{}~\begin{quote}\begin{description}
\item[{Parameters}] \leavevmode
\textbf{\texttt{x\_new}} (\emph{list}) -- a \(k\)-vector from the domain

\item[{Returns}] \leavevmode
the predicted benefit in f\_min of sampling the objective function at x\_new

\item[{Return type}] \leavevmode
float

\end{description}\end{quote}

\end{fulllineitems}

\index{max\_likelihood() (smbo.models.dace method)}

\begin{fulllineitems}
\phantomsection\label{index:smbo.models.dace.max_likelihood}\pysiglinewithargsret{\bfcode{max\_likelihood}}{\emph{bounds=None}, \emph{verbose=False}}{}~\begin{quote}\begin{description}
\item[{Parameters}] \leavevmode
\textbf{\texttt{bounds}} (\emph{list}) -- the \(P  imes Q\) domain over which likelihood is being maximized, defaults to \((1,2)       imes(0,\infty)\)

\item[{Returns}] \leavevmode
res: an object describing the \(P\) and :math{}`Q{}` values that optimize likelihood

\item[{Return type}] \leavevmode
optimization\_result

\end{description}\end{quote}

The evaluation of this function also resets self.P and self.Q to the values indicated by res, i.e.
it sets P and Q to maximize the likelihood of the DACE model, thereby fitting the model to the data.

\end{fulllineitems}

\index{pred\_err() (smbo.models.dace method)}

\begin{fulllineitems}
\phantomsection\label{index:smbo.models.dace.pred_err}\pysiglinewithargsret{\bfcode{pred\_err}}{\emph{x\_new}}{}~\begin{quote}\begin{description}
\item[{Parameters}] \leavevmode
\textbf{\texttt{x\_new}} (\emph{list}) -- a \(k\)-vector from the domain

\item[{Returns}] \leavevmode
the predicted function value at x\_new

\item[{Return type}] \leavevmode
float

\end{description}\end{quote}

This is computed using the so-called best linear unbiased predictor,  Jones Eq. 7. Variables such as the correlation matrix :math:'mathbb\{R\}' are saved as they are copmuted lazily by this and other methods.

\end{fulllineitems}

\index{predict() (smbo.models.dace method)}

\begin{fulllineitems}
\phantomsection\label{index:smbo.models.dace.predict}\pysiglinewithargsret{\bfcode{predict}}{\emph{x\_new}}{}~\begin{quote}\begin{description}
\item[{Parameters}] \leavevmode
\textbf{\texttt{x\_new}} (\emph{list}) -- a \(k\)-vector from the domain

\item[{Returns}] \leavevmode
the predicted function value at x\_new

\item[{Return type}] \leavevmode
float

\end{description}\end{quote}

This is computed using the so-called best linear unbiased predictor,  Jones Eq. 7

\end{fulllineitems}

\index{reset() (smbo.models.dace method)}

\begin{fulllineitems}
\phantomsection\label{index:smbo.models.dace.reset}\pysiglinewithargsret{\bfcode{reset}}{}{}
Resets all lazyprops

\end{fulllineitems}


\end{fulllineitems}

\index{dace\_function() (in module smbo.models)}

\begin{fulllineitems}
\phantomsection\label{index:smbo.models.dace_function}\pysiglinewithargsret{\code{smbo.models.}\bfcode{dace\_function}}{\emph{X}, \emph{Y}}{}~\begin{quote}\begin{description}
\item[{Parameters}] \leavevmode\begin{itemize}
\item {} 
\textbf{\texttt{X}} (\emph{list}) -- a list of input vectors

\item {} 
\textbf{\texttt{Y}} (\emph{list}) -- a list of observed objective values

\end{itemize}

\item[{Returns}] \leavevmode
(pred\_y,pred\_err): two functions, each k-to-1, where k is the dimension of the input space, representing the dace predictor surface and predicted error at each point in input space

\item[{Return type}] \leavevmode
tuple

\end{description}\end{quote}

This instantiates a dace class member behind the scenes and returns its predictor function, and the predicted error function of its predictor function.

\end{fulllineitems}

\phantomsection\label{index:module-smbo.samplers}\index{smbo.samplers (module)}\phantomsection\label{index:module-samplers}\index{samplers (module)}\index{latin\_hypercube() (in module smbo.samplers)}

\begin{fulllineitems}
\phantomsection\label{index:smbo.samplers.latin_hypercube}\pysiglinewithargsret{\code{smbo.samplers.}\bfcode{latin\_hypercube}}{\emph{m}, \emph{k}, \emph{bounds=None}, \emph{rand\_sampler=\textless{}built-in method random of Random object\textgreater{}}}{}~\begin{quote}\begin{description}
\item[{Parameters}] \leavevmode\begin{itemize}
\item {} 
\textbf{\texttt{m}} (\emph{int}) -- the number of desired sample points

\item {} 
\textbf{\texttt{k}} (\emph{int}) -- the dimension of input space

\item {} 
\textbf{\texttt{bounds}} (\emph{list}) -- the \(k\) min-max tuples describing the function domain as a \(k\)-rectangle. Defaults to the unit \(k\)-cube.

\item {} 
\textbf{\texttt{rand\_sampler}} (\emph{function}) -- used to choose actual sample coordinates once the latin hypercube selects a sample's particular hyper(sub)rectangle in the input domain.

\end{itemize}

\item[{Returns}] \leavevmode
An \(m\)-list of \(k\)-vectors, representing an \(m\)-point latin hypercube sample of the \(k\)-dimensional input domain.

\item[{Return type}] \leavevmode
list

\end{description}\end{quote}

\end{fulllineitems}

\phantomsection\label{index:module-smbo.lazyprop}\index{smbo.lazyprop (module)}\phantomsection\label{index:module-lazyprop}\index{lazyprop (module)}\index{lazyprop() (in module smbo.lazyprop)}

\begin{fulllineitems}
\phantomsection\label{index:smbo.lazyprop.lazyprop}\pysiglinewithargsret{\code{smbo.lazyprop.}\bfcode{lazyprop}}{\emph{fn}}{}~\begin{quote}\begin{description}
\item[{Parameters}] \leavevmode
\textbf{\texttt{fn}} (\emph{function}) -- a function, whose only argument is self, whose output shouldn't
be computed more than once for a given X,Y pair.

\item[{Returns}] \leavevmode
\begin{description}
\item[{\_lazyprop: a function that calls fn the first time it is called, then remembers}] \leavevmode
that output and returns this remembered value after subsequent calls

\end{description}


\item[{Return type}] \leavevmode
function

\end{description}\end{quote}

\end{fulllineitems}

\index{reset\_lps() (in module smbo.lazyprop)}

\begin{fulllineitems}
\phantomsection\label{index:smbo.lazyprop.reset_lps}\pysiglinewithargsret{\code{smbo.lazyprop.}\bfcode{reset\_lps}}{\emph{self}}{}
Resets all lazyprops, so that the evaluated function vals are forgotten and must be
recomputed from raw data

\end{fulllineitems}



\section{License}
\label{index:license}
The project is licensed under the MIT license.


\renewcommand{\indexname}{Python Module Index}
\begin{theindex}
\def\bigletter#1{{\Large\sffamily#1}\nopagebreak\vspace{1mm}}
\bigletter{l}
\item {\texttt{lazyprop}} \emph{(Unix, Windows)}, \pageref{index:module-lazyprop}
\indexspace
\bigletter{m}
\item {\texttt{models}} \emph{(Unix, Windows)}, \pageref{index:module-models}
\indexspace
\bigletter{s}
\item {\texttt{samplers}} \emph{(Unix, Windows)}, \pageref{index:module-samplers}
\item {\texttt{smb\_optimizer}} \emph{(Unix, Windows)}, \pageref{index:module-smb_optimizer}
\item {\texttt{smbo}}, \pageref{index:module-smbo}
\item {\texttt{smbo.lazyprop}}, \pageref{index:module-smbo.lazyprop}
\item {\texttt{smbo.models}}, \pageref{index:module-smbo.models}
\item {\texttt{smbo.samplers}}, \pageref{index:module-smbo.samplers}
\item {\texttt{smbo.smb\_optimizer}}, \pageref{index:module-smbo.smb_optimizer}
\end{theindex}

\renewcommand{\indexname}{Index}
\printindex
\end{document}
